\documentclass[a4paper,11pt]{article}
\usepackage{cv}
\name{Andreas Christian M\"uller}
\info{Address: &Department of Computer Science, University of Bonn, Germany\\
      Phone: & +49 (0)228 73 4175\\
      Email: & amueller@ais.uni-bonn.de\\
      WWW: & www.ais.uni-bonn.de/~amueller/}
 
\bibliography{cv}
 
\addtocategory{conferences}{mueller2010topological,schulz2010exploiting, scherer2010evaluation, muller2012information}
\addtocategory{workshops}{schulzinvestigating, muellermulti}
\addtocategory{papers}{schulz2011exploiting}

\begin{document}

\maketitle

\section{Education and Qualifications}
\begin{tabular}{lll}
    2009 & Diploma in Mathematics \\ & University of Bonn, Grade A-\\
         & Topic: ``Singularities of Minimal Degenerations in Affine Grassmannians'' \\
    expected October 2013 & PhD in Computer Science \\ & University of Bonn \\
         & Topic: ``Structured Prediction Models for Semantic Image Segmentation''
\end{tabular}

\section{Current Position}
\begin{tabular}{lll}
    since 2010 & PhD Student at the Department of Computer Science, University of Bonn, Germany\\
               & Advisor: Prof. Sven Behnke\\
    since 2010 & PhD Scholarship of the B-IT, Bonn/Aachen, Germany\\
               & renewed 2011 until 2013\\
    starting October 2013& Machine Learning Scientist at Amazon Research, Berlin, Germany
\end{tabular}

\section{Other Positions}
\begin{tabular}{lll}
    2011 & Lecture Assistant at the Department of Computer Science, University of Bonn, Germany \\
    spring 2012 & Visiting Scientist at the Austrian Institute of Science and Technology\\
               & Host: Prof. Christoph Lampert\\
    summer 2012 & Research Intern at Microsoft Research Cambridge\\
               & Hosts: Carsten Rother, Sebastian Nowozin\\
\end{tabular}


\section{Past Research}
\begin{itemize}
    \item Generative models of image patches and images.
    \item Connectionist and deep models.
    \item Kernel methods for multi-instance learning.
    \item Non-parametric entropy estimates for unsupervised learning.
    \item Random forests for structured output spaces.
\end{itemize}

\section{Current Research and Thesis}
\begin{itemize}
    \item Conditional random fields for image processing and segmentation.
    \item Inference and learning in CRFs / structured SVMs.
    \item Latent variable CRFs.
\end{itemize}

\section{Open Source Projects}
\begin{itemize}
    \item Maintainer and core developer for the Python machine learning package ``scikit-learn''\footnote{\url{http://scikit-learn.org/}}.
    \item Creator and maintainer Python package ``pystruct''\footnote{\url{http://pystruct.github.org/}} for structured prediction.
    \item Co-author of ``CUV'', a C++ and Python interface for CUDA,
        targeted at machine learning and computer vision.\footnote{\url{https://github.com/deeplearningais/CUV}}
    \item Contributor to the Python computer vision package ``scikit-image''\footnote{\url{http://scikit-image.org/}}.
\end{itemize}
\pagebreak

\section{Peer Reviewing}
\begin{itemize}
    \item Journal of Machine Learning Research, open source software track
    \item Journal of Pattern Analysis and Machine Intelligence
    \item European Conference of Computer Vision
\end{itemize}

\section{Spoken Languages}
\begin{itemize}
    \item German: Native.
    \item English: Full professional proficiency.
    \item French: Elementary proficiency.
\end{itemize}

\section{Programming Languages}
\begin{itemize}
    \item C++ (C++03 and C++11): Strong knowledge.
    \item Python / Cython: Very strong knowledge, in particular for scientific programming.
    \item CUDA (with C++): Good knowledge.
    \item Java: Basic knowledge.
\end{itemize}

\begin{publications}
    \printbib{papers}
    \printbib{conferences}
    \printbib{workshops}
\end{publications}

\end{document}
