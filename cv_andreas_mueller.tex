\documentclass[a4paper,11pt]{article}
\usepackage{cv}
\name{Andreas Christian M\"uller}
\info{Address: &108 W 116th Street, New York, NY\\
      Email: & \href{mailto:importamueller+cv@gmail.com}{importamueller+cv@gmail.com}\\
      Web: & \url{http://amueller.io}\\}
 
\bibliography{cv}
 
\addtocategory{conferences}{mueller2010topological,schulz2010exploiting, scherer2010evaluation, muller2012information, muellericra}
\addtocategory{workshops}{schulzinvestigating, muellermulti, buitinck2013api}
\addtocategory{papers}{schulz2011exploiting, muellerpystruct, varoquaux2015scikit, abraham2014machine, severin2017computational, huppenkothen2016using}
\addtocategory{books}{amuellerintropython}

\begin{document}

\maketitle

\section{Education and Qualifications}
\begin{tabular}{lll}
    2009 & Diploma in Mathematics,  University of Bonn\\
         & Thesis: ``Singularities of Minimal Degenerations in Affine Grassmannians'' \\
    2014 & PhD in Computer Science, University of Bonn \\
         & Thesis: ``Methods for Learning Structured Prediction in Semantic Segmentation''
\end{tabular}

\section{Current Position}
\begin{tabular}{lll}
    Since 2018 & \textbf{Associate Research Scientist at Columbia University}\\
               & Teaching in the Data Science Master program, \\
               & scikit-learn development and various research activities.
\end{tabular}

\section{Past Positions}
\begin{tabular}{lll}

    2010--2013 & \textbf{PhD Student at the Department of Computer Science, University of Bonn, Germany}\\
         & Advisor: Prof.\ Sven Behnke. \\
    2010--2013 & PhD Scholarship of the B-IT, Bonn/Aachen, Germany\\
    2011 and 2013& Lecture Assistant at the Department of Computer Science, University of Bonn, Germany \\
    Spring 2012 & \textbf{Visiting Scientist at the Austrian Institute of Science and Technology}\\
               & Host: Prof.\ Christoph Lampert\\
    Summer 2012 & \textbf{Research Intern at Microsoft Research Cambridge}\\
               & Hosts: Carsten Rother, Sebastian Nowozin\\
    2013--2014 & \textbf{Machine Learning Scientist at Amazon Development Center Germany}\\
              & Design and implementation of large-scale machine learning and\\
              & computer vision applications.\\
    2014--2016 & \textbf{Research Engineer at the NYU Center for Data Science}\\
               & Development of open source tools for machine learning and data science.\\
    2016--2018 & \textbf{Lecturer in Discipline at Columbia University}\\
               & Teaching in the Data Science Master program, \\
               & scikit-learn development and various research activities.
\end{tabular}

\section{Awarded Grants}
\begin{itemize}
    \item \emph{Scikit-learn maintenance and enhancement to gradient boosting and search} (PI). Chan-Zuckerberg Initiative \$150k. 2019-2020.
    \item \emph{Extension \& Maintenance of Scikit-learn} (PI). Alfred P. Sloan Foundation. \$313k. 2017-2019.
    \item \emph{Analysis and Extension of Scikit-learn} (PI). Bloomberg. \$63k. 2017-2018.
    \item \emph{SI2-SSE: Improving Scikit-learn usability and automation} (PI). NSF. \$400k. 2017-2020.
    \item \emph{Big Data Map and Assets Platform (BDMAP) Phase I - Collaborative Resource and Understanding eXchange (CRUX)} (senior personel, project lead). NSF. \$100k. 2017-2018.
    \item \emph{Building blocks and Search Improvements for Automated Machine Learning Model Selection} (PI). DARPA. \$351k. 2018.
\end{itemize}

\pagebreak

\section{Open Source Contributions}
\begin{itemize}
    \item Core developer and member of the Technical Committee for the Python machine learning package ``scikit-learn''\footnote{\url{http://scikit-learn.org/}}.
    \item Creator and maintainer of the Python package ``PyStruct''\footnote{\url{http://pystruct.github.org/}} for structured prediction.
    \item Co-author of ``CUV'', a C++ and Python interface for CUDA,
        targeted at deep learning.\footnote{\url{https://github.com/deeplearningais/CUV}}
    \item Contributor to the Python computer vision package ``scikit-image''\footnote{\url{http://scikit-image.org/}}.
\end{itemize}

\section{Professional Activities}
\subsection{Journal Editorial Board}
\begin{itemize}
    \item Action Editor, Journal of Machine Learning Research, OSS Track
\end{itemize}
\subsection{Journal and Converence Reviewing}
\begin{itemize}
    \item Nature
    \item Neural Information Processing System
    \item International Conference of Machine Learning
    \item European Conference of Computer Vision
    \item Journal of Statistical Software
    \item Journal of Machine Learning Research
    \item Journal of Pattern Analysis and Machine Intelligence
\end{itemize}
\subsection{Postdoctoral Fellows}
\begin{itemize}
    \item Jan van Rijn
    \item Nicolas Hug
\end{itemize}
\subsection{Advising and Consulting}
\begin{itemize}
    \item Scientific Advisor, Life Epigenetics
    \item Scientific Advisor, Ocean Protocol Foundation Ltd
    \item Advisory board, Scikit-learn @ Inria Foundation
\end{itemize}
\begin{publications}
    \printbib{books}
    \printbib{papers}
    \printbib{conferences}
    \printbib{workshops}
\end{publications}

\end{document}
