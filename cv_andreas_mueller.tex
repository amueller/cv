\documentclass[a4paper,11pt]{article}
\usepackage{cv}
\name{Andreas Christian M\"uller}
\info{Address: &156 East 2nd Street, New York, NY\\
      Phone: & +1 347 697 3364\\
      Email: & importamueller@gmail.com}
 
\bibliography{cv}
 
\addtocategory{conferences}{mueller2010topological,schulz2010exploiting, scherer2010evaluation, muller2012information, muellericra}
\addtocategory{workshops}{schulzinvestigating, muellermulti, buitinck2013api}
\addtocategory{papers}{schulz2011exploiting, muellerpystruct, varoquaux2015scikit, abraham2014machine}
\addtocategory{books}{amuellerintropython}

\begin{document}

\maketitle

\section{Education and Qualifications}
\begin{tabular}{lll}
    2009 & Diploma in Mathematics \\ & University of Bonn\\
         & Thesis: ``Singularities of Minimal Degenerations in Affine Grassmannians'' \\
    2014 & PhD in Computer Science \\ & University of Bonn \\
         & Thesis: ``Methods for Learning Structured Prediction in Semantic Segmentation''
\end{tabular}

\section{Current Position}
\begin{tabular}{lll}
    Since 2016 & \textbf{Lecturer in Data Science at Columbia University}\\
               & Teaching in the Data Science Master program, leading a data science helpdesk.\\
\end{tabular}

\section{Past Positions}
\begin{tabular}{lll}
    2014--2016 & \textbf{Research Engineer at the NYU Center for Data Science}\\
               & Development of open source tools for machine learning and data science.\\
    2010--2013 & \textbf{PhD Student at the Department of Computer Science, University of Bonn, Germany}\\
         & Advisor: Prof.\ Sven Behnke. \\
    2010--2013 & PhD Scholarship of the B-IT, Bonn/Aachen, Germany\\
    2011 and 2013& Lecture Assistant at the Department of Computer Science, University of Bonn, Germany \\
    Spring 2012 & \textbf{Visiting Scientist at the Austrian Institute of Science and Technology}\\
               & Host: Prof.\ Christoph Lampert\\
    Summer 2012 & \textbf{Research Intern at Microsoft Research Cambridge}\\
               & Hosts: Carsten Rother, Sebastian Nowozin\\
    2013--2014 & \textbf{Machine Learning Scientist at Amazon Development Center Germany}\\
              & Design and implementation of large-scale machine learning and\\
              & computer vision applications.
\end{tabular}

\section{Open Source Contributions}
\begin{itemize}
    \item Maintainer and core developer for the Python machine learning package ``scikit-learn''\footnote{\url{http://scikit-learn.org/}}.
    \item Creator and maintainer of the Python package ``PyStruct''\footnote{\url{http://pystruct.github.org/}} for structured prediction.
    \item Co-author of ``CUV'', a C++ and Python interface for CUDA,
        targeted at machine learning and computer vision.\footnote{\url{https://github.com/deeplearningais/CUV}}
    \item Contributor to the Python computer vision package ``scikit-image''\footnote{\url{http://scikit-image.org/}}.
\end{itemize}
\pagebreak

\section{Peer Reviewing}
\begin{itemize}
    \item Neural Information Processing System
    \item Journal of Machine Learning Research
    \item Journal of Pattern Analysis and Machine Intelligence
    \item European Conference of Computer Vision
\end{itemize}

\section{Spoken Languages}
\begin{itemize}
    \item German: Native.
    \item English: Full professional proficiency.
    \item French: Elementary proficiency.
\end{itemize}

\section{Programming Languages}
\begin{itemize}
    \item Python / Cython: Very strong knowledge, in particular for scientific programming.
    \item C++ (C++03 and C++11): Strong knowledge.
    \item CUDA (with C++): Good knowledge.
    \item Java: Basic knowledge.
    \item Scala: Basic knowledge.
\end{itemize}

\begin{publications}
    \printbib{books}
    \printbib{papers}
    \printbib{conferences}
    \printbib{workshops}
\end{publications}

\end{document}
